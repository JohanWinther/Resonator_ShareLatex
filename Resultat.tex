% CREATED BY DAVID FRISK, 2016
\documentclass[main.tex]{subfiles}
 
\begin{document}

\chapter{Resultat}
\label{ch:results}
I följande avsnitt presenteras först mätningarna av $S_{21}$ för ett antal resonatorer

\section{Jämförelse av intern kvalitetsfaktor utan och med filter}
%I \figref{fig:qi_pwr} visas $Q_i$ över antalet genomsnittliga fotoner i 
%\tikzfig{matningar/AlSi-C_5.732_GHz_Qi_photons}{resonator $f_r = \unit[5.732]{GHz} $. Legend saknas}{fig:qi_pwr}{\textwidth}{20em}
Jämförelse av $Q_i$ med och utan filter görs genom att plotta $Q_i$ mot genomsnittliga antalet fotoner i resonatorn. I det här avsnittet presenteras ett urval sådana figurer. Övriga mätningar finns i appendix \ref{app:data}.

Samtliga figurer innehåller en kurvanpassning utifrån modellen \ref{ekv:TLSmodel} och värdet på parametrarna är presenterade i figurtexterna.

Mätningar på prov A visade blandade resultat. \Figref{fig:A4.953} visar exempel på en resonator där Q-värdet förändrades väldigt lite efter våra filter hade kopplats på. \Figref{fig:A5.006} och \figref{fig:A5.689} visar resonatorer vilkas Q-värde har förbättras respektive försämrats över hela effektintervallet. Samtliga av dessa mätningar från prov A passade väl med anpassningen till TLS-modellen \ref{ekv:TLSmodel}.



\begin{figure}
    \begin{subfigure}{0.5\textwidth}
    \centering
    \setlength\figurewidth{0.8\linewidth}
    \setlength\figureheight{11em}
    % This file was created by matlab2tikz.
%
\definecolor{mycolor1}{rgb}{0.12527,0.32424,0.83027}%
\definecolor{mycolor2}{rgb}{0.99904,0.76531,0.21641}%
%
\begin{tikzpicture}[%
trim axis left, trim axis right
]

\begin{axis}[%
width=0.96\figurewidth,
height=\figureheight,
at={(0\figurewidth,0\figureheight)},
scale only axis,
bar shift auto,
xmin=0,
xmax=7,
xtick={1,2,3,4,5,6},
xticklabels={{5,15},{5,37},{6,13},{6,44},{7,57},{7,84}},
xticklabel style = {font=\footnotesize},
xlabel style={font=\color{white!15!black}},
xlabel={$f_r$ (\unit{GHz})},
ymin=0,
ymax=3e-07,
ylabel style={font=\color{white!15!black}},
ylabel={$1/Q_o$},
yticklabel style={font=\footnotesize},
axis background/.style={fill=white},
title style={font=\bfseries},
title={},
legend style={legend cell align=left, align=left, draw=white!15!black,legend pos = north west}
]
\addplot[ybar, bar width=0.229, fill=mycolor1, draw=black, area legend] table[row sep=crcr] {%
1	6.30549618735363e-08\\
2	7.01023767596739e-08\\
3	6.34400611286079e-08\\
4	3.77130813762326e-08\\
5	1.15668525313106e-07\\
6	2.52305686010039e-07\\
};
\addplot[forget plot, color=white!15!black] table[row sep=crcr] {%
0	0\\
7	0\\
};
\addlegendentry{utan filter}

\addplot[ybar, bar width=0.229, fill=mycolor2, draw=black, area legend] table[row sep=crcr] {%
1	4.08124644535775e-08\\
2	4.11019336724848e-08\\
3	1.42139113974824e-08\\
4	4.26629500372091e-08\\
5	1.75105525406755e-07\\
6	2.1613786228122e-07\\
};
\addplot[forget plot, color=white!15!black] table[row sep=crcr] {%
0	0\\
7	0\\
};
\addlegendentry{med filter}

\end{axis}
\end{tikzpicture}%
    \caption{C}
    \end{subfigure}
    \begin{subfigure}{0.5\textwidth}
    \centering
    \setlength\figurewidth{0.8\linewidth}
    \setlength\figureheight{11em}
    % This file was created by matlab2tikz.
%
\definecolor{mycolor1}{rgb}{0.12527,0.32424,0.83027}%
\definecolor{mycolor2}{rgb}{0.99904,0.76531,0.21641}%
%
\begin{tikzpicture}[%
trim axis left, trim axis right
]

\begin{axis}[%
width=0.96\figurewidth,
height=\figureheight,
at={(0\figurewidth,0\figureheight)},
scale only axis,
bar shift auto,
xmin=0,
xmax=7,
xtick={1,2,3,4,5,6},
xticklabels={{4,82},{5,01},{5,73},{6,03},{7,08},{7,32}},
xticklabel style={font=\footnotesize},
xlabel style={font=\color{white!15!black}},
xlabel={$f_r$ (\unit{GHz})},
ymin=0,
ymax=1.5e-06,
ylabel style={font=\color{white!15!black}},
ylabel={$1/Q_o$},
yticklabel style={font=\footnotesize},
axis background/.style={fill=white},
title style={font=\bfseries},
title={},
legend style={at={(0.03,0.97)}, anchor=north west, legend cell align=left, align=left, draw=white!15!black}
]
\addplot[ybar, bar width=0.229, fill=mycolor1, draw=black, area legend] table[row sep=crcr] {%
1	4.58419825830351e-07\\
2	4.91937737970534e-07\\
3	5.54169809853124e-07\\
4	4.62732133455839e-07\\
5	6.74115628511589e-07\\
6	8.77602589651056e-07\\
};
\addplot[forget plot, color=white!15!black] table[row sep=crcr] {%
0	0\\
7	0\\
};
\addlegendentry{utan filter}

\addplot[ybar, bar width=0.229, fill=mycolor2, draw=black, area legend] table[row sep=crcr] {%
1	3.04198848676838e-07\\
2	3.50049432143065e-07\\
3	3.28035188151382e-07\\
4	2.87052605969869e-07\\
5	4.74508917723777e-07\\
6	1.46200417792788e-06\\
};
\addplot[forget plot, color=white!15!black] table[row sep=crcr] {%
0	0\\
7	0\\
};
\addlegendentry{med filter}

\end{axis}
\end{tikzpicture}%
    \caption{d}
    \end{subfigure}
    \begin{center}
    \begin{subfigure}{0.75\textwidth}
    \centering
    \setlength\figurewidth{0.8\linewidth}
    \setlength\figureheight{11em}
    % This file was created by matlab2tikz.
%
\definecolor{mycolor1}{rgb}{0.12527,0.32424,0.83027}%
\definecolor{mycolor2}{rgb}{0.99904,0.76531,0.21641}%
%
\begin{tikzpicture}[%
trim axis left, trim axis right
]

\begin{axis}[%
width=0.96\figurewidth,
height=\figureheight,
at={(0\figurewidth,0\figureheight)},
scale only axis,
bar shift auto,
xmin=0,
xmax=10,
xtick={1,2,3,4,5,6,7,8,9},
xticklabels={{4.953},{5.006},{5.218},{5.689},{5.959},{6.251},{6.946},{7.351},{7.624}},
xlabel style={font=\color{white!15!black}},
xlabel={Resonansfrekvens (\unit{GHz})},
ymin=0,
ymax=1e-05,
ylabel style={font=\color{white!15!black}},
ylabel={$1/Q_o$},
axis background/.style={fill=white},
title style={font=\bfseries},
title={},
axis x line*=bottom,
axis y line*=left,
legend style={legend cell align=left, align=left, draw=white!15!black}
]
\addplot[ybar, bar width=0.229, fill=mycolor1, draw=black, area legend] table[row sep=crcr] {%
1	4.17784145538606e-07\\
2	5.54828668926099e-06\\
3	6.32830955730862e-07\\
4	2.93833437172275e-06\\
5	1.69718196196557e-06\\
8	2.53328393751367e-06\\
6	8.82972087457609e-06\\
7	4.03277602782519e-06\\
9	5.07823900963456e-07\\
};
\addplot[forget plot, color=white!15!black] table[row sep=crcr] {%
0	0\\
10	0\\
};
\addlegendentry{utan filter}

\addplot[ybar, bar width=0.229, fill=mycolor2, draw=black, area legend] table[row sep=crcr] {%
1	4.26920066540144e-07\\
2	5.33980650200332e-06\\
3	7.03137022021399e-07\\
4	3.91309025572688e-06\\
5	1.51479763666507e-06\\
8	2.78274475976737e-06\\
6	9.50500678222247e-06\\
7	3.51241501320627e-06\\
9	3.46547848189305e-07\\
};
\addplot[forget plot, color=white!15!black] table[row sep=crcr] {%
0	0\\
10	0\\
};
\addlegendentry{med filter}

\end{axis}
\end{tikzpicture}%
    \caption{A}
    \end{subfigure}
    \end{center}
    \caption{Caption}
    \label{fig:my_label}
\end{figure}

\begin{figure}[H]
  \centering
  \setlength\figurewidth{0.8\textwidth}
  \setlength\figureheight{15em}
  % This file was created by matlab2tikz.
%
\definecolor{mycolor1}{rgb}{0.9990 0.7653 0.2164}%
\definecolor{mycolor2}{rgb}{0.1253 0.3242 0.8303}%
%
\begin{tikzpicture}[%
trim axis left, trim axis right
]

\begin{axis}[%
width=0.953\figurewidth,
height=\figureheight,
at={(0\figurewidth,0\figureheight)},
scale only axis,
xmode=log,
xmin=0.01,
xmax=1000000,
xminorticks=true,
xlabel style={font=\color{white!15!black}},
xlabel={$\text{\textless{}n\textgreater}$},
ymin=400000,
ymax=2200000,
ylabel style={font=\color{white!15!black}},
ylabel={Qi},
axis background/.style={fill=white},
legend style={legend cell align=left, align=left, draw=white!15!black,font=\tiny,legend pos = north west}
]
\addplot [color=mycolor1, draw=none, mark=square, mark options={solid, mycolor1}]
  table[row sep=crcr]{%
0.122710685117133	466620.795299\\
0.222200142995475	460287.515639\\
0.376431494230062	484339.806259\\
0.668109433520876	493979.800591\\
1.17072140892931	507809.550376\\
1.95404059439818	540470.816246\\
3.41727699460125	550114.681419\\
5.5861631818117	611810.879719\\
9.4970199980524	640507.052643\\
15.3913591867746	713320.323811\\
26.3575360048921	736413.537167\\
30.422525995812	822160.225818\\
42.3433111344225	820049.974628\\
52.0985417719463	870161.21041\\
67.9534638743783	919681.69385\\
82.5626779949239	957001.067043\\
111.869757736073	989476.96591\\
132.958790608764	1051294.03267\\
183.42908970699	1076398.62626\\
218.832923527528	1132415.04563\\
376.916523183775	1178829.47158\\
617.998254615749	1273097.90972\\
1055.07143034399	1335123.06703\\
1759.74764845591	1401008.79135\\
3025.71363494939	1476114.43283\\
4980.6922361558	1555348.16019\\
8571.68665844077	1616169.29499\\
14325.0510349027	1696625.81447\\
24453.5991136159	1763845.28223\\
40868.3686425512	1832508.31889\\
79401.7483785842	1864567.22016\\
121427.148010041	1918901.82907\\
208569.917715575	1986700.5916\\
364398.38770411	2032515.05863\\
620699.530793056	2085844.75535\\
};
\addlegendentry{med filter}

\addplot [color=mycolor1]
  table[row sep=crcr]{%
0.122710685117133	462729.121965341\\
0.222200142995475	467707.52563091\\
0.376431494230062	475036.47864261\\
0.668109433520876	487785.632175579\\
1.17072140892931	507054.255610615\\
1.95404059439818	532194.530367619\\
3.41727699460125	569070.346203638\\
5.5861631818117	609911.043279011\\
9.4970199980524	662394.658201218\\
15.3913591867746	716892.072074661\\
26.3575360048921	784044.389217259\\
30.422525995812	802938.433822764\\
42.3433111344225	847869.219946717\\
52.0985417719463	876926.596162284\\
82.5626779949239	943509.229771391\\
132.958790608764	1014788.62338068\\
218.832923527528	1091110.62576728\\
1055.07143034399	1335024.20587627\\
1759.74764845591	1412538.66366883\\
3025.71363494939	1492349.45304172\\
4980.6922361558	1563046.76081443\\
8571.68665844077	1636553.24962131\\
14325.0510349027	1702316.20447501\\
24453.5991136159	1766586.60043264\\
40868.3686425512	1824080.3879609\\
79401.7483785842	1892160.06400975\\
121427.148010041	1932004.70538437\\
208569.917715575	1978657.856039\\
364398.38770411	2022148.55697909\\
620699.530793056	2059495.91849287\\
};
\addlegendentry{anpassad med filter}

\addplot [color=mycolor2, draw=none, mark=square, mark options={solid, mycolor2}]
  table[row sep=crcr]{%
0.0365162078911759	431794.097437\\
0.0632398357543756	431213.298127\\
0.112676979827531	428491.776266\\
0.197688118403027	450941.543167\\
0.357905997789356	443442.907327\\
0.611377673170861	453115.570884\\
1.01793839558865	496452.786387\\
1.71158492372239	522886.605327\\
2.84494457411741	570792.861668\\
4.76685587336918	612633.908963\\
8.16785111406262	636801.287109\\
13.2328856145823	705681.257263\\
21.7256339325379	772594.478224\\
35.4423766797641	849531.73749\\
59.926867527558	894211.544317\\
97.6231534413402	981252.09571\\
162.533369433665	1050050.84649\\
287.76041816455	1119763.64867\\
453.582048220754	1195616.8372\\
762.125029130598	1263234.74921\\
1277.449813474	1331298.62132\\
2153.2650953331	1403733.0719\\
3584.71084427416	1486367.93769\\
6015.73661882288	1557701.44817\\
10054.0915080187	1640175.72755\\
17116.997259868	1710635.77605\\
33045.4653260758	1769360.54122\\
49827.0172130848	1837821.49573\\
84555.9346831575	1891942.98839\\
146607.678589366	1944527.00807\\
247249.675438732	1998433.80933\\
426839.577673579	2049135.40123\\
766908.20721948	2081260.83751\\
};
\addlegendentry{utan filter}

\addplot [color=mycolor2]
  table[row sep=crcr]{%
0.0365162078715991	474609.639587372\\
0.0632398357543756	475439.636137459\\
0.112676979887939	476960.915808828\\
0.19768811850901	479535.065432506\\
0.357905997981234	484249.754114802\\
0.611377673170861	491372.173481323\\
1.01793839558865	502037.214458698\\
1.71158492372239	518446.267107195\\
2.8449445725922	541566.006891133\\
4.76685587336918	573608.78466117\\
8.16785111406262	616999.356604438\\
13.2328856145823	664407.920985457\\
21.7256339441852	720829.069397695\\
35.442376660763	783208.709775088\\
59.926867527558	856310.593692757\\
97.6231534413402	928821.374853501\\
162.533369520801	1008169.50494701\\
287.760418318822	1100125.90443991\\
2153.26509648749	1429245.1696574\\
3584.71084427416	1509305.81851465\\
6015.73662204799	1587696.87850909\\
10054.0915080187	1661948.86247492\\
17116.9972506913	1734672.30232807\\
33045.4653260758	1818119.47645398\\
49827.0172130848	1866416.53368092\\
84555.934637826	1924224.75313173\\
146607.678589366	1979155.59182701\\
247249.675438732	2026494.98427671\\
426839.577902413	2071125.74012632\\
766908.20721948	2113835.42766977\\
};
\addlegendentry{anpassad utan filter}

\end{axis}
\end{tikzpicture}%
  \caption{AlSi-A\_4.953\_GHz, Intern Q-faktor för resonator med och utan filter vid olika antal fotoner i resonatorn. Mätningarna är gjorda vid resonansfrekvensen \unit[4,953]{GHz}. Anpassning till \ref{ekv:TLSmodel} gav $Q_{\text{annat}}^{\text{filter}}=2,342\cdot10^6$, $Q_{\text{annat}}^{\text{utan}}=2,394\cdot10^6$.} 
  \label{fig:A4.953}
\end{figure}

\begin{figure}[H]
  \centering
  \setlength\figurewidth{0.8\textwidth}
  \setlength\figureheight{15em}
  % This file was created by matlab2tikz.
%
\definecolor{mycolor1}{rgb}{0.85000,0.32500,0.09800}%
\definecolor{mycolor2}{rgb}{0.00000,0.44700,0.74100}%
%
\begin{tikzpicture}[%
trim axis left, trim axis right
]

\begin{axis}[%
width=0.953\figurewidth,
height=\figureheight,
at={(0\figurewidth,0\figureheight)},
scale only axis,
xmode=log,
xmin=0.01,
xmax=1000000,
xminorticks=true,
xlabel style={font=\color{white!15!black}},
xlabel={$\text{\textless{}n\textgreater}$},
ymin=130000,
ymax=190000,
ylabel style={font=\color{white!15!black}},
ylabel={Qi},
axis background/.style={fill=white},
legend style={legend cell align=left, align=left, draw=white!15!black}
]
\addplot [color=mycolor1, draw=none, mark=square, mark options={solid, mycolor1}]
  table[row sep=crcr]{%
0.113634480893775	141665.377916\\
0.20106495566147	143351.270083\\
0.348051892691989	147944.968603\\
0.616385852851424	149948.986389\\
1.08561083759663	151113.881792\\
1.92349132741203	153062.995374\\
3.38174884899031	155822.329959\\
5.96783137400924	157299.120074\\
10.4743719221365	160606.54121\\
18.3682222012236	163802.9645\\
32.197410355299	167428.158069\\
40.5518476868123	166389.735386\\
56.5039335994624	169140.350116\\
71.5475427713422	169376.676318\\
99.4146851008304	171261.406593\\
124.427663373087	172321.785569\\
174.133606746283	173367.056133\\
219.595691020234	173539.402686\\
305.509504570165	176196.475009\\
384.819460989901	175627.344353\\
676.584374858532	178211.274817\\
1195.10945946391	179350.39194\\
2106.03951214842	180640.643759\\
3730.78694846042	181797.284133\\
6575.42903677984	182788.942712\\
11704.9098450474	183657.362214\\
20590.524486644	184022.787214\\
36326.2687240655	184476.286598\\
63958.3257739663	184888.323508\\
113441.657322243	184991.761665\\
211635.384503589	184893.162153\\
356057.539518612	184786.017531\\
};
\addlegendentry{with filter}

\addplot [color=mycolor1]
  table[row sep=crcr]{%
0.113634480893775	145363.689200036\\
0.20106495566147	145829.961909831\\
0.348051892691989	146556.199934839\\
0.616385852851424	147727.62879424\\
1.08561083759663	149418.734639286\\
1.92349132741203	151707.408765748\\
3.38174884899031	154459.053425246\\
5.96783137400924	157543.0788341\\
10.4743719221365	160697.341421403\\
18.3682222012236	163770.489082544\\
32.197410355299	166650.835248538\\
40.5518476868123	167764.202836058\\
56.5039335994624	169286.683938755\\
71.5475427713422	170312.335756413\\
99.4146851008304	171661.292525025\\
124.427663373087	172528.644554016\\
174.133606746283	173749.263552573\\
219.595691020234	174538.630015528\\
305.509504570165	175590.688836223\\
384.819460989901	176278.253879978\\
676.584374858532	177804.707803366\\
1195.10945946391	179140.818472252\\
2106.03951214842	180292.128516789\\
3730.78694846042	181294.800739908\\
6575.42903677984	182150.613633758\\
11704.9098450474	182898.920097702\\
20590.524486644	183527.993386655\\
36326.2687240655	184070.750134859\\
63958.3257739663	184534.492767795\\
113441.657322243	184936.710704289\\
211635.384503589	185308.276950544\\
356057.539518612	185573.044103079\\
};
\addlegendentry{fit with filter}

\addplot [color=mycolor2, draw=none, mark=square, mark options={solid, mycolor2}]
  table[row sep=crcr]{%
0.0375241091678652	131699.232457\\
0.0643568949846857	143497.663519\\
0.11497416688218	142759.727919\\
0.207641006145348	138162.825415\\
0.36774117606915	137158.239703\\
0.659434339274006	138119.532895\\
1.17030370745213	137252.093777\\
2.04304634043841	141947.392891\\
3.58974446656814	145891.358118\\
6.33200126920161	148006.155729\\
11.1150913360378	149950.938023\\
19.4150461984816	154699.796378\\
34.1738740568966	156686.670679\\
59.6735445750161	161233.392042\\
104.884451941669	163716.56416\\
184.567010704762	165868.593975\\
324.02338016308	168460.357173\\
579.348050290254	170506.786109\\
1005.74876890764	172452.020522\\
1767.21140392309	173655.595296\\
3115.56165997346	174673.12748\\
5486.38132137491	175754.439695\\
9602.92906930528	176742.448341\\
16954.530993885	177335.123879\\
29659.7450921302	178335.976639\\
52468.0969994201	178596.980788\\
99937.7047320937	178233.956867\\
164767.462836713	178264.216752\\
289409.845880347	178692.786799\\
};
\addlegendentry{no filter}

\addplot [color=mycolor2]
  table[row sep=crcr]{%
0.0375241091678652	137865.367616437\\
0.0643568949846857	137934.697356585\\
0.11497416688218	138064.115338287\\
0.207641006145348	138296.54691986\\
0.36774117606915	138685.053675027\\
0.659434339274006	139353.76159935\\
1.17030370745213	140419.068824951\\
2.04304634043841	141989.303752471\\
3.58974446656814	144220.255377799\\
6.33200126920161	147109.46269202\\
11.1150913360378	150451.522373538\\
19.4150461984816	153981.593311072\\
34.1738740568966	157525.066649934\\
59.6735445750161	160799.377592065\\
104.884451941669	163786.80960609\\
184.567010704762	166414.181013616\\
324.02338016308	168670.122457506\\
579.348050290254	170649.733558917\\
1005.74876890764	172235.60176699\\
1767.21140392309	173599.063287518\\
3115.56165997346	174745.17755882\\
5486.38132137491	175697.478476789\\
9602.92906930528	176480.768975425\\
16954.530993885	177140.55898837\\
29659.7450921302	177678.130976315\\
52468.0969994201	178131.544774552\\
99937.7047320937	178548.742784667\\
164767.462836713	178815.048705047\\
289409.845880347	179065.400448334\\
};
\addlegendentry{ fit no filter}

\end{axis}
\end{tikzpicture}%
  \caption{AlSi-A\_5.006\_GHz Intern Q-faktor för resonator med och utan filter vid olika antal fotoner i resonatorn. Mätningarna är gjorda vid resonansfrekvensen \unit[4,953]{GHz}. Anpassning till \ref{ekv:TLSmodel} gav $Q_{\text{annat}}^{\text{filter}}=1,873\cdot10^5$, $Q_{\text{annat}}^{\text{utan}}=1,802\cdot10^5$.}
  \label{fig:A5.006}
\end{figure}

\begin{figure}[H]
  \centering
  \setlength\figurewidth{0.8\textwidth}
  \setlength\figureheight{15em}
  % This file was created by matlab2tikz.
%
\definecolor{mycolor1}{rgb}{0.9990 0.7653 0.2164}%
\definecolor{mycolor2}{rgb}{0.1253 0.3242 0.8303}%
%
\begin{tikzpicture}[%
trim axis left, trim axis right
]

\begin{axis}[%
width=0.953\figurewidth,
height=\figureheight,
at={(0\figurewidth,0\figureheight)},
scale only axis,
xmode=log,
xmin=0.01,
xmax=1000000,
xminorticks=true,
xlabel style={font=\color{white!15!black}},
xlabel={$\text{\textless{}n\textgreater}$},
ymin=160000,
ymax=340000,
ylabel style={font=\color{white!15!black}},
ylabel={Qi},
axis background/.style={fill=white},
legend style={legend cell align=left, align=left, draw=white!15!black,font=\tiny,legend pos = south east}
]
\addplot [color=mycolor1, draw=none, mark=square, mark options={solid, mycolor1}]
  table[row sep=crcr]{%
0.11790518987484	174075.228243\\
0.206141078952283	183483.134062\\
0.368538204331717	182388.934112\\
0.658612222515732	179354.032123\\
1.1539758070184	188808.389838\\
2.02224020175944	188856.286012\\
3.49204506040338	198121.19506\\
6.17582800545808	199040.145708\\
10.7169642075312	208315.93804\\
18.8271015773455	211429.144999\\
32.8067128436663	216825.097296\\
40.9367595303706	217091.310877\\
56.9847421686971	222561.639288\\
72.7877936997125	217698.336637\\
100.027371249678	226323.369488\\
125.315678663872	226079.528075\\
173.375681814175	232340.622509\\
219.456557731918	231859.05316\\
303.925982628447	236485.637027\\
384.466104699091	234560.717313\\
668.349640855792	238376.204669\\
1179.08002970097	241715.735508\\
2046.16330341778	243560.504123\\
3659.91394276282	245795.269104\\
6382.27414013425	248314.695374\\
11339.8026849621	250317.142997\\
19942.182397947	251861.161184\\
35154.8028692217	253000.360647\\
60987.2962862832	254044.354096\\
108162.695300916	255055.149143\\
206314.651795315	254615.890263\\
334849.070061705	254687.898617\\
};
%\addlegendentry{with filter}

\addplot [color=mycolor1]
  table[row sep=crcr]{%
0.11790518987484	180402.374319056\\
0.206141078952283	180905.501797474\\
0.368538204331717	181791.618846033\\
0.658612222515732	183259.418160108\\
1.1539758070184	185481.878419195\\
2.02224020175944	188725.419248068\\
3.49204506040338	192957.674437467\\
6.17582800545808	198372.803528627\\
10.7169642075312	204239.329158716\\
18.8271015773455	210444.543089325\\
32.8067128436663	216393.124745561\\
40.9367595303706	218658.159006021\\
56.9847421686971	221894.843979982\\
72.7877936997125	224164.92638742\\
100.027371249678	226944.790104625\\
125.315678663872	228797.974013088\\
173.375681814175	231295.489213164\\
219.456557731918	232984.053705143\\
303.925982628447	235149.896973094\\
384.466104699091	236597.63552141\\
668.349640855792	239645.425302358\\
1179.08002970097	242300.089345549\\
2046.16330341778	244475.73883645\\
3659.91394276282	246400.916009214\\
6382.27414013425	247938.653786314\\
11339.8026849621	249264.337494671\\
19942.182397947	250346.284829798\\
35154.8028692217	251248.844285164\\
60987.2962862832	251977.920410399\\
108162.695300916	252607.02616133\\
206314.651795315	253185.621909106\\
334849.070061705	253544.252272518\\
};
%\addlegendentry{fit with filter}

\addplot [color=mycolor2, draw=none, mark=square, mark options={solid, mycolor2}]
  table[row sep=crcr]{%
0.0282055251781551	205658.481002\\
0.0494194388596904	214703.62581\\
0.0902597782758664	208532.924557\\
0.154532368898038	219932.403792\\
0.276466058308513	224840.870182\\
0.479072795728832	228241.572837\\
0.842670222482719	232475.902478\\
1.47513805176047	238003.738583\\
2.59723881895823	240718.244384\\
4.55812738550613	244650.011027\\
7.9362576621881	251107.968676\\
13.7860399544492	259134.891791\\
24.0447472945439	265470.518198\\
41.6810963947065	273768.299154\\
72.2640546879434	282346.136884\\
125.552034174246	289131.942528\\
219.822161058284	293606.951044\\
390.203021573965	299897.203067\\
672.97799596149	304605.493201\\
1176.0333117148	310090.361668\\
2058.15204624911	314542.20056\\
3584.33807762132	319492.191655\\
6266.46195207164	322715.09325\\
10992.1204006125	325762.694326\\
19230.6376258202	328152.141548\\
33852.1961123483	330105.02521\\
64314.0717283457	331347.286749\\
105917.113556082	332225.255039\\
187860.475110678	332285.507021\\
331760.984239548	331845.737338\\
};
%\addlegendentry{no filter}

\addplot [color=mycolor2]
  table[row sep=crcr]{%
0.0282055251781551	215292.427698488\\
0.0494194388596904	215644.163821532\\
0.0902597782758664	216303.987585975\\
0.154532368898038	217299.126849977\\
0.276466058308513	219057.078183533\\
0.479072795728832	221663.035565692\\
0.842670222482719	225593.295673646\\
1.47513805176047	230908.893695937\\
2.59723881895823	237630.70332042\\
4.55812738550613	245324.068342479\\
7.9362576621881	253443.205938171\\
13.7860399544492	261634.859378193\\
24.0447472945439	269672.224726868\\
41.6810963947065	277204.052068066\\
72.2640546879434	284208.675734238\\
125.552034174246	290656.164289089\\
219.822161058284	296584.900802141\\
390.203021573965	302039.145332032\\
672.97799596149	306668.080771057\\
1176.0333117148	310889.770743345\\
2058.15204624911	314637.705833288\\
3584.33807762132	317916.768987072\\
6266.46195207164	320822.682604992\\
10992.1204006125	323386.385799135\\
19230.6376258202	325618.871257565\\
33852.1961123483	327588.930160192\\
64314.0717283457	329517.660746767\\
105917.113556082	330818.790816141\\
187860.475110678	332126.082057215\\
331760.984239548	333249.569961296\\
};
%\addlegendentry{ fit no filter}

\end{axis}
\end{tikzpicture}%
  \caption{AlSi-A\_5.689\_GHz Intern Q-faktor för resonator med och utan filter vid olika antal fotoner i resonatorn. Mätningarna är gjorda vid resonansfrekvensen \unit[4,953]{GHz}. Anpassning till \ref{ekv:TLSmodel} gav  $Q_{\text{annat}}^{\text{filter}}=2,556\cdot10^5$, $Q_{\text{annat}}^{\text{utan}}=3,403\cdot10^5$.}
  \label{fig:A5.689}
\end{figure}


För prov C skedde inga försämringar till följd av filtrena. \figref{fig:C6.436} visar ett exempel på en resonator från prov C där filtrena gav upphov till en liten men konsekvent förbättring av $Q_i$

\begin{figure}[H]
  \centering
  \setlength\figurewidth{0.8\textwidth}
  \setlength\figureheight{15em}
  % This file was created by matlab2tikz.
%
\definecolor{mycolor1}{rgb}{0.9990 0.7653 0.2164}%
\definecolor{mycolor2}{rgb}{0.1253 0.3242 0.8303}%
%
\begin{tikzpicture}[%
trim axis left, trim axis right
]

\begin{axis}[%
width=0.953\figurewidth,
height=\figureheight,
at={(0\figurewidth,0\figureheight)},
scale only axis,
xmode=log,
xmin=0.01,
xmax=1000000,
xminorticks=true,
xlabel style={font=\color{white!15!black}},
xlabel={$\text{\textless{}n\textgreater}$},
ymin=300000,
ymax=1200000,
ylabel style={font=\color{white!15!black}},
ylabel={Qi},
axis background/.style={fill=white},
legend style={at={(0.03,0.97)}, anchor=north west, legend cell align=left, align=left, draw=white!15!black}
]
\addplot [color=mycolor1, draw=none, mark=square, mark options={solid, mycolor1}]
  table[row sep=crcr]{%
0.13276073045954	348054.803758\\
0.226789996045743	349793.978726\\
0.403818318016465	338775.828966\\
0.689909828507124	370979.730043\\
1.25379476651749	382618.808674\\
2.07542949232333	387053.973378\\
3.57891305621537	400039.722909\\
6.09098005002001	414307.061566\\
10.4295829733019	431389.439545\\
17.8579177118993	445619.515521\\
30.3206908508585	470054.35095\\
38.5099490487839	444019.640621\\
51.6628410368411	479031.310163\\
75.8838099500939	452330.991752\\
88.8574424802387	496372.366708\\
111.1551921315	481141.255444\\
149.588749048147	521973.131851\\
194.125249794957	509622.729621\\
255.821118416977	546253.713155\\
327.671844577872	533467.50164\\
537.465126148443	559566.940339\\
954.330400869664	581838.730901\\
1550.65979508135	616030.155608\\
2710.98019030831	651849.910271\\
4422.17444954707	693474.323998\\
7766.3193024544	738419.074127\\
12837.8324876924	788475.171417\\
21521.190325612	840671.042753\\
35832.109430771	891984.863234\\
60433.214082208	951927.212447\\
122116.850673547	1035439.9317\\
169004.001482331	1102437.68758\\
281384.825928056	1176006.75414\\
};
%\addlegendentry{with filter}

\addplot [color=mycolor1]
  table[row sep=crcr]{%
0.13276073045954	371090.373902059\\
0.226789996045743	371497.834856622\\
0.403818318016465	372255.332774944\\
0.689909828507124	373453.871510376\\
1.25379476651749	375729.245739737\\
2.07542949232333	378857.905654277\\
3.57891305621537	384094.56094773\\
6.09098005002001	391735.166813907\\
10.4295829733019	402630.077287429\\
17.8579177118993	417079.583080513\\
30.3206908508585	434798.356168029\\
38.5099490487839	443877.554109181\\
51.6628410368411	455888.652279023\\
75.8838099500939	472925.855469738\\
88.8574424802387	480324.655927281\\
111.1551921315	491200.736296836\\
149.588749048147	506281.683969707\\
194.125249794957	520102.391910435\\
255.821118416977	535308.503575128\\
327.671844577872	549430.671097836\\
537.465126148443	578991.388336747\\
954.330400869664	615462.493984229\\
1550.65979508135	648108.965144903\\
2710.98019030831	687754.576082353\\
4422.17444954707	724343.997104338\\
7766.3193024544	768667.991533325\\
12837.8324876924	810285.200667771\\
21521.190325612	855154.913512011\\
35832.109430771	901572.191886094\\
60433.214082208	951439.286345109\\
122116.850673547	1022309.20397744\\
169004.001482331	1056543.92404599\\
281384.825928056	1112211.15778432\\
};
%\addlegendentry{fit with filter}

\addplot [color=mycolor2, draw=none, mark=square, mark options={solid, mycolor2}]
  table[row sep=crcr]{%
0.10860412939107	321265.856329\\
0.171227347599518	312098.17975\\
0.272039183714063	301581.895599\\
0.426841350454082	310075.583844\\
0.649563152344378	332376.312052\\
1.02256733749516	336206.711714\\
1.59490580524859	342898.338962\\
2.5028817255838	339977.010009\\
3.88567474006405	346233.028261\\
6.02748737903448	351074.988484\\
9.21594882117464	368160.265911\\
14.2126501904667	375891.715754\\
21.5136297319087	397105.29254\\
35.6288042945291	405314.641883\\
51.2444867730431	420722.647216\\
78.4346108809908	432264.446777\\
120.239898289744	443423.314522\\
183.651152117652	461853.767527\\
282.402589723542	479311.989883\\
433.650584732042	491312.922188\\
658.529490302256	510848.228606\\
1007.55798696953	533319.131281\\
1538.28352585481	557964.539599\\
2307.96376210048	584893.251452\\
4056.9260572495	611105.676776\\
5371.27570602861	641687.535661\\
8037.21873257482	675113.41427\\
12222.3432435847	707037.686088\\
18218.2465095594	741162.930934\\
27717.073551544	776879.204248\\
41995.6049097187	808902.006614\\
64801.2794041913	846424.363547\\
98021.7851458341	899069.555024\\
147643.553458624	941643.02449\\
215679.788852234	979789.587722\\
408843.288497746	1009208.77824\\
};
%\addlegendentry{no filter}

\addplot [color=mycolor2]
  table[row sep=crcr]{%
0.10860412939107	325324.631341804\\
0.171227347599518	325615.857583081\\
0.272039183714063	326080.119077035\\
0.426841350454082	326782.356595209\\
0.649563152344378	327770.918305591\\
1.02256733749516	329372.292315645\\
1.59490580524859	331708.687846648\\
2.5028817255838	335152.598184947\\
3.88567474006405	339884.563651541\\
6.02748737903448	346268.858688227\\
9.21594882117464	354246.488655041\\
14.2126501904667	364326.32968864\\
21.5136297319087	375798.935801114\\
35.6288042945291	392028.242293104\\
51.2444867730431	405132.718105822\\
78.4346108809908	421837.775980087\\
120.239898289744	439950.317594563\\
183.651152117652	459138.130632024\\
282.402589723542	479818.492498187\\
433.650584732042	501583.9105952\\
658.529490302256	523875.768730699\\
1007.55798696953	547677.945357209\\
1538.28352585481	572488.454245794\\
2307.96376210048	597357.55553558\\
4056.9260572495	633750.618051479\\
5371.27570602861	652670.85143333\\
8037.21873257482	680820.68652212\\
12222.3432435847	711364.757500666\\
18218.2465095594	741693.679714054\\
27717.073551544	774936.665215006\\
41995.6049097187	809279.93057911\\
64801.2794041913	846705.262182237\\
98021.7851458341	883972.713428286\\
147643.553458624	922415.239742407\\
215679.788852234	959416.803555986\\
408843.288497746	1025116.98109019\\
};
%\addlegendentry{ fit no filter}

\end{axis}
\end{tikzpicture}%
  \caption{AlSi-C\_6.436\_GHz Intern Q-faktor för resonator med och utan filter vid olika antal fotoner i resonatorn. Mätningarna är gjorda vid resonansfrekvensen \unit[4,953]{GHz}. Anpassning till \ref{ekv:TLSmodel} gav $Q_{\text{annat}}^{\text{filter}}=8,727\cdot10^6$, $Q_{\text{annat}}^{\text{utan}}=1,565\cdot10^6$.}
  \label{fig:C6.436}
\end{figure}

De flesta graferna från mätningarna på d-provet hade ett utseende som liknar det hos \figref{fig:d6.025}, alltså ett någorlunda oförändrat Q-värde under ett visst antal fotoner och en ökande förbättring efter det.

\begin{figure}[H]
  \centering
  \setlength\figurewidth{0.8\textwidth}
  \setlength\figureheight{15em}
  % This file was created by matlab2tikz.
%
\definecolor{mycolor1}{rgb}{0.85000,0.32500,0.09800}%
\definecolor{mycolor2}{rgb}{0.00000,0.44700,0.74100}%
%
\begin{tikzpicture}[%
trim axis left, trim axis right
]

\begin{axis}[%
width=0.953\figurewidth,
height=\figureheight,
at={(0\figurewidth,0\figureheight)},
scale only axis,
xmode=log,
xmin=0.01,
xmax=1000000,
xminorticks=true,
xlabel style={font=\color{white!15!black}},
xlabel={$\text{\textless{}n\textgreater}$},
ymin=200000,
ymax=2200000,
ylabel style={font=\color{white!15!black}},
ylabel={Qi},
axis background/.style={fill=white},
legend style={at={(0.03,0.97)}, anchor=north west, legend cell align=left, align=left, draw=white!15!black}
]
\addplot [color=mycolor1, draw=none, mark=square, mark options={solid, mycolor1}]
  table[row sep=crcr]{%
0.164836873549653	237529.660277\\
0.279153189995739	250660.953191\\
0.445034332932342	292870.534493\\
0.834419954295998	306190.100355\\
1.40460441297846	295302.260248\\
2.44906711230343	300337.592656\\
4.06160518212671	327130.326712\\
6.78389587148024	354576.570552\\
11.1503674922052	387868.325183\\
18.5451254941127	416024.213184\\
22.0714777927035	442535.562499\\
30.1733163908002	455714.872732\\
39.6284505732501	476582.543836\\
48.0700393127517	511366.405466\\
58.1693865721458	530437.419592\\
76.6498624767217	564938.477407\\
92.5218594785534	589341.478036\\
123.454942997831	623882.551411\\
146.084376305374	663603.791791\\
231.288048454788	750197.302604\\
366.105613272187	845279.70943\\
568.225933417993	971943.588618\\
902.30032744148	1086693.7818\\
1443.30463774401	1210915.70522\\
2289.52261473925	1349529.48291\\
3677.22907366075	1487945.50582\\
5932.0623973286	1620018.11386\\
9567.7056829425	1754154.7523\\
15565.2649103318	1886946.18121\\
32534.9867328793	2014331.46869\\
};
\addlegendentry{with filter}

\addplot [color=mycolor1]
  table[row sep=crcr]{%
0.164836873549653	261068.828766942\\
0.279153189995739	263088.636579519\\
0.445034332932342	265953.806642219\\
0.834419954295998	272397.766800234\\
1.40460441297846	281207.119435224\\
2.44906711230343	295773.021383659\\
4.06160518212671	315262.821861807\\
6.78389587148024	342569.317382446\\
11.1503674922052	377328.272884403\\
18.5451254941127	422076.865576074\\
22.0714777927035	439591.048268922\\
30.1733163908002	473912.571768735\\
39.6284505732501	506856.004745438\\
48.0700393127517	531904.997375394\\
58.1693865721458	558036.256366821\\
76.6498624767217	598295.975931366\\
92.5218594785534	627424.004789525\\
123.454942997831	674679.621026952\\
146.084376305374	703710.307608735\\
231.288048454788	788370.764242098\\
366.105613272187	880739.773649475\\
568.225933417993	976121.238630129\\
902.30032744148	1083348.44839981\\
1443.30463774401	1198831.00229429\\
2289.52261473925	1317831.74954476\\
3677.22907366075	1444672.74037223\\
5932.0623973286	1576137.27706485\\
15565.2649103318	1845735.02445055\\
32534.9867328793	2049187.76672961\\
};
\addlegendentry{fit with filter}

\addplot [color=mycolor2, draw=none, mark=square, mark options={solid, mycolor2}]
  table[row sep=crcr]{%
0.102044500362575	226427.339695\\
0.180113336872952	202742.864592\\
0.265257584992582	217936.236432\\
0.337419598183369	288832.037074\\
0.515689395385321	300863.057711\\
0.781513219147657	317992.501312\\
1.16223523797189	341248.444154\\
1.75752890356045	360496.040517\\
2.65805156523926	380775.426862\\
3.94802026929113	408356.325645\\
6.0236477592508	426909.265859\\
9.13627517014158	443718.61475\\
14.1498762816808	471183.912904\\
20.2758301426376	509947.340626\\
30.3492687037984	541485.008634\\
44.6825611102083	584088.654971\\
65.8915073733155	628689.546473\\
94.1218337606992	696802.668693\\
136.40461851212	762845.387503\\
197.926433498246	831980.146312\\
294.636823139001	883990.349088\\
430.60244708335	955932.810645\\
634.622128744831	1020912.9629\\
1019.23938389807	1105051.72538\\
1383.9745723329	1176881.90664\\
2071.88610731806	1247483.63395\\
3123.17127206821	1302527.1126\\
4704.48915205956	1362465.9645\\
6883.92315088392	1464683.65111\\
10398.990033502	1526899.22621\\
15662.4783356747	1592614.28465\\
23810.7902237696	1639194.15672\\
36567.5466492666	1684290.04438\\
52913.2548321763	1734599.60606\\
102387.056884331	1777177.74257\\
125253.399409118	1807669.89753\\
198873.245619956	1841989.72124\\
316147.785764525	1880144.28211\\
};
\addlegendentry{no filter}

\addplot [color=mycolor2]
  table[row sep=crcr]{%
0.102044500362575	265400.301435983\\
0.180113336872952	268599.350396083\\
0.265257584992582	271974.924464026\\
0.337419598183369	274749.528476052\\
0.515689395385321	281295.500684461\\
0.781513219147657	290344.694794994\\
1.16223523797189	302088.13037494\\
1.75752890356045	318235.479991502\\
2.65805156523926	338971.949874388\\
3.94802026929113	363479.815372022\\
6.0236477592508	394909.0683911\\
9.13627517014158	431231.496598913\\
14.1498762816808	474932.830804586\\
20.2758301426376	514952.583160313\\
30.3492687037984	563962.059745419\\
44.6825611102083	614800.35614741\\
65.8915073733155	669324.528837513\\
94.1218337606992	722151.818207789\\
136.40461851212	779614.597393005\\
197.926433498246	839462.801599641\\
294.636823139001	905398.469751595\\
430.60244708335	969710.989780852\\
2071.88610731806	1240014.27712431\\
3123.17127206821	1308809.02250691\\
4704.48915205956	1375618.91613714\\
6883.92315088392	1435609.12352618\\
10398.990033502	1497953.92358609\\
15662.4783356747	1556785.09977216\\
23810.7902237696	1613533.93791584\\
36567.5466492666	1667888.76762596\\
52913.2548321763	1711565.77065964\\
102387.056884331	1782343.66021723\\
125253.399409118	1802128.72217073\\
198873.245619956	1844372.00399942\\
316147.785764525	1882507.22937519\\
};
\addlegendentry{ fit no filter}

\end{axis}
\end{tikzpicture}%
  \caption{AlSi-d\_6.025\_GHz Intern Q-faktor för resonator med och utan filter vid olika antal fotoner i resonatorn. Mätningarna är gjorda vid resonansfrekvensen \unit[6,025]{GHz}. Anpassning till \ref{ekv:TLSmodel} gav $Q_{\text{annat}}^{\text{filter}}=1,036\cdot10^7$, $Q_{\text{annat}}^{\text{utan}}=2,229\cdot10^7$.}
  \label{fig:d6.025}
\end{figure}

\end{document}