% CREATED BY DAVID FRISK, 2016
\documentclass[main.tex]{subfiles}
 
\begin{document}

\chapter{Resultat}
\label{sec:results}
I det här avsnittet presenteras resultaten från experimenten som beskrevs i kapitel \ref{ch:exp}. 

I tabell \ref{tab:Q_iA} presenteras $Q_i$ och procentuella förändringen av $Q_i$ för våra prover vid $\expval{n}=1$  och $\expval{n}=10^4$, före och efter våra filter monterades. Här är $\expval{n}$ det genomsnittliga antalet fotoner i resonatorn vid resonans. \figref{fig:Q_in} visar mätresultatet för en resonator där filtrena åstadkom en förbättring vid höga effekter. 

I \figref{fig:Q_in} syns även den kurvanspassning till modellen \eqref{ekv:TLSmodel} som gjordes för samtliga mätningar. Vid en bra anpassning ger anpassningsparametern $\frac{1}{Q_o}$ information om förluster som inte är TLS-relaterade, det vill säga bland annat de förluster relaterade till brutna Cooperpar. Hur de övriga förlusterna förändrades efter implementering av våra filter presenteras i \figref{fig:Q_other_C}, \figref{fig:Q_other_d} och \figref{fig:Q_other_A}. Eftersom $\frac{1}{Q_o}$ betecknar övriga förluster är en minskning med filter ett positivt resultat. 

%\section{Jämförelse av intern kvalitetsfaktor utan och med filter}
%I \figref{fig:qi_pwr} visas $Q_i$ över antalet genomsnittliga fotoner i 
%\tikzfig{matningar/AlSi-C_5.732_GHz_Qi_photons}{resonator $f_r = \unit[5.732]{GHz} $. Legend saknas}{fig:qi_pwr}{\textwidth}{20em}
%Jämförelse av $Q_i$ med och utan filter görs genom att plotta $Q_i$ mot genomsnittliga antalet fotoner i resonatorn. I det här avsnittet presenteras ett urval sådana figurer. Övriga mätningar finns i appendix \ref{app:data}.

%Samtliga figurer innehåller en kurvanpassning utifrån modellen \ref{ekv:TLSmodel} och värdet på parametrarna är presenterade i figurtexterna.


%Mätningar på prov A visade blandade resultat. \Figref{fig:A4.953} visar exempel på en resonator där Q-värdet förändrades väldigt lite efter våra filter hade kopplats på. \Figref{fig:A5.006} och \figref{fig:A5.689} visar resonatorer vilkas Q-värde har förbättras respektive försämrats över hela effektintervallet. Samtliga av dessa mätningar från prov A passade väl med anpassningen till TLS-modellen \ref{ekv:TLSmodel}.


\begin{figure}[H]
  \centering
  \setlength\figurewidth{0.75\textwidth}
  \setlength\figureheight{12em}
  % This file was created by matlab2tikz.
%
\definecolor{mycolor1}{rgb}{0.9990 0.7653 0.2164}%
\definecolor{mycolor2}{rgb}{0.1253 0.3242 0.8303}%
%
\begin{tikzpicture}[%
trim axis left, trim axis right
]

\begin{axis}[%
width=0.953\figurewidth,
height=\figureheight,
at={(0\figurewidth,0\figureheight)},
scale only axis,
xmode=log,
xmin=0.1,
xmax=100000,
xminorticks=true,
xlabel style={font=\color{white!15!black}},
xlabel={$\langle n\rangle$},
ymin=0,
ymax=2500000,
ylabel style={font=\color{white!15!black}},
ylabel={Qi},
axis background/.style={fill=white},
legend style={at={(0.03,0.97)}, anchor=north west, legend cell align=left, align=left, draw=white!15!black}
]
\addplot [color=mycolor1, draw=none, mark=square, mark options={solid, mycolor1}]
  table[row sep=crcr]{%
0.112842475476945	257324.789315\\
0.166950197177758	324658.040512\\
0.306222515846021	310513.97963\\
0.582378003532025	315020.343051\\
0.877863068097618	349692.369116\\
1.4202516018218	389708.680774\\
2.46377084519998	400983.090608\\
4.29353559445227	409696.743212\\
6.99658912879994	454407.455982\\
11.6937191341621	485470.052523\\
13.0614066946906	553309.511619\\
18.27016536608	556249.513637\\
21.5823589080574	635835.821758\\
29.3677101094807	617902.503363\\
32.9426973550036	698727.056839\\
44.4828194254173	724250.484397\\
51.5036406505989	787212.374469\\
69.6963739659049	821712.390837\\
80.1773406007741	899558.244637\\
123.550563377404	1047663.93957\\
194.112215199968	1176892.98597\\
311.946948992423	1312947.97489\\
494.001909501612	1463364.42238\\
805.527395110918	1600517.53508\\
1296.16561727461	1748607.09616\\
2124.98964748589	1887611.00193\\
3486.93743911246	2011946.76665\\
5797.40655345243	2148696.34929\\
9693.50293091667	2276666.51368\\
18875.5039603376	2415307.25845\\
};
\addlegendentry{med filter}

\addplot [color=mycolor1]
  table[row sep=crcr]{%
0.112842475476945	304967.307418869\\
0.166950197177758	306780.72254\\
0.306222515846021	311362.236321297\\
0.582378003532025	320104.637189592\\
0.877863068097618	329004.272727193\\
1.4202516018218	344284.963273088\\
2.46377084519998	370596.184187966\\
4.29353559445227	409695.69747409\\
6.99658912879994	456814.200750075\\
11.6937191341621	520948.552223138\\
13.0614066946906	536822.536009057\\
18.27016536608	589568.132494476\\
21.5823589080574	618326.454270763\\
29.3677101094807	675978.491682348\\
32.9426973550036	698955.911889852\\
44.4828194254173	762744.889775284\\
51.5036406505989	795779.034097283\\
69.6963739659049	867766.901376138\\
80.1773406007741	902771.784206255\\
123.550563377404	1016961.48861988\\
194.112215199968	1144994.28675888\\
311.946948992423	1286941.85988017\\
494.001909501612	1429376.67271369\\
1296.16561727461	1731909.81616461\\
2124.98964748589	1882506.49204471\\
3486.93743911246	2026407.76694997\\
5797.40655345243	2164407.12208679\\
9693.50293091667	2292112.27431043\\
18875.5039603376	2438311.85977707\\
};
\addlegendentry{anpassad med filter}

\addplot [color=mycolor2, draw=none, mark=square, mark options={solid, mycolor2}]
  table[row sep=crcr]{%
0.143351117274636	288170.933638\\
0.226914262025544	299608.283496\\
0.343903752358678	304086.786502\\
0.578054791255249	280217.024837\\
0.842783822955178	314921.30046\\
1.3864643106188	298928.335786\\
2.13871154498994	308188.842152\\
3.18399712978659	334289.618141\\
4.50596258002778	384282.101076\\
6.83621253327378	403033.751418\\
9.88071692817047	450309.172499\\
14.6078719923256	486046.322104\\
21.988281828123	532089.325524\\
30.7538343373594	594064.306761\\
44.2741872837558	658594.423241\\
64.2001633524047	723838.86846\\
90.7142516496516	818209.473664\\
131.92980606354	894354.909926\\
195.147463640319	955381.988197\\
286.144067621112	1024691.46657\\
418.488520000407	1104294.29975\\
616.032213522856	1180670.99783\\
915.921318502799	1245842.66046\\
1536.09916144067	1306291.51991\\
2076.20331036623	1368239.15953\\
3137.72336206425	1424195.91746\\
4704.00522467627	1477659.78237\\
7208.23972025754	1521743.44079\\
10717.4918137142	1579532.67031\\
16226.1027964	1619854.38172\\
25111.2667113541	1652189.6098\\
39479.9387136244	1692493.48931\\
59351.028865955	1725884.04431\\
};
\addlegendentry{utan filter}

\addplot [color=mycolor2]
  table[row sep=crcr]{%
0.143351117274636	290730.545232927\\
0.226914262025544	292802.188400545\\
0.343903752358678	295659.262748629\\
0.578054791255249	301233.36369953\\
0.842783822955178	307318.860082749\\
1.3864643106188	319170.000891989\\
2.13871154498994	334331.348805166\\
3.18399712978659	353451.178330347\\
4.50596258002778	375069.453673592\\
6.83621253327378	408007.814450752\\
9.88071692817047	444046.893609245\\
14.6078719923256	489710.796938791\\
21.988281828123	545549.234205014\\
30.7538343373594	597169.965710064\\
44.2741872837558	658553.86407172\\
64.2001633524047	725996.431304142\\
90.7142516496516	792184.29687864\\
131.92980606354	866477.385809564\\
286.144067621112	1022981.19933039\\
418.488520000407	1098520.44022987\\
616.032213522856	1172780.27524682\\
915.921318502799	1245175.34997869\\
1536.09916144067	1332407.00666952\\
2076.20331036623	1379047.1681489\\
3137.72336206425	1437664.64506205\\
4704.00522467627	1489114.53052036\\
7208.23972025754	1537022.91819175\\
10717.4918137142	1576010.85292212\\
16226.1027964	1611480.66117444\\
25111.2667113541	1643503.55714857\\
39479.9387136244	1671554.65275685\\
59351.028865955	1692904.30728776\\
};
\addlegendentry{anpassad utan filter}

\end{axis}
\end{tikzpicture}%
  \caption{$Q_i$ mot genomsnittligt antal fotoner i resonatorn $(f_r=\unit[5,732]{GHz})$ för prov 2. Figuren visar mätningar med och utan det medellånga filtret. Mätdatan är anpassad till TLS-modellen \eqref{ekv:TLSmodel}. Vid $\expval{n}<10^1$ uppvisar resonatorn samma $Q_i$ med och utan filter. För $\expval{n}>10^3$ finns en tydlig förbättring av $Q_i$ med filtret, som även ökar för större $\expval{n}$.}
  \label{fig:Qi_n}
\end{figure}

I tabell \ref{tab:Qi_A} presenteras $Q_i$ vid $\expval{n}=1$ och $\expval{n}=10^4$ för alla resonatorer, med och utan filter. Vi kan

\begin{table}[h]
\centering
\caption{$Q_i$ vid låg ($\expval{n}=1$) och hög effekt ($\expval{n}=10^4$) för alla resonatorer, med och utan filter. Notera att majoriteten av resonatorerna i prov 1 och 2 har en positiv förbättring vid hög effekt, där alla förbättringar i prov 2 är över $10 \%$.}
\label{tab:Qi_A}


\begin{tabular}{crrrrrrr}
\toprule
 & $f_r$ (\unit{GHz}) & låg $Q_i^{\text{utan}}$ & låg $Q_i^{\text{filter}}$ &  \% & hög $Q_i^{\text{utan}}$ & hög $Q_i^{\text{filter}}$ &  \% \\
\cmidrule{2-8}

\parbox[t]{2mm}{\multirow{6}{*}{\rotatebox[origin=c]{90}{Prov 1}}}
& 5,153 & 357991 & 365356 & +2 & 763255 & 781503 & +2 \\
 & 5,366 & 345178 & 349090 & +1 & 786969 & 793972 & +1 \\ 
 & 6,132 & 334621 & 329145 & -2 & 717688 & 719756 & +0 \\ 
 & 6,436 & 335975 & 377380 & +12 & 690086 & 760466 & +10 \\ 
 & 7,569 & 261340 & 274584 & +5 & 408291 & 437707 & +7 \\
 & 7,844 & 179143 & 207177 & +16 & 249860 & 295340 & +18\\

\cmidrule{2-8}
\parbox[t]{2mm}{\multirow{6}{*}{\rotatebox[origin=c]{90}{Prov 2}}}
& 4,822 & 366937 & 391124 & +7 & 1838027 & 2443362 & +33 \\
& 5,012 & 346153 & 368569 & +6 & 1602518 & 2052125 & +28 \\
& 5,732 & 310297 & 358703 & +16 & 1567717 & 2281294 & +46 \\
& 6,025 & 331339 & 303028 & -9 & 1519837 & 1763726 & +16 \\
& 7,082 & 361850 & 296749 & -18 & 1167356 & 1295680 & +11 \\
& 7,322 & 274503 & 228240 & -17 & 972577 & 521539 & -46 \\

\cmidrule{2-8}
\parbox[t]{2mm}{\multirow{9}{*}{\rotatebox[origin=c]{90}{Prov 3}}}
& 4,953 & 494541 & 503112 & +2 & 1639071 & 1636143 & +0 \\
& 5,006 & 137541 & 150901 & +10 & 176774 & 183369 & +4 \\
& 5,218 & 473506 & 446376 & -6 & 1225300 & 1111122 & -9 \\
& 5,689 & 233851 & 185870 & -21 & 325123 & 249776 & -23 \\
& 5,959 & 320237 & 323121 & +1 & 547830 & 557199 & +2 \\
& 6,251 & 232915 & 237031 & +2 & 368657 & 330500 & -10 \\
& 6,946 & 100658 & 88755 & -12 & 113367 & 102477 & -10 \\
& 7,351 & 182866 & 206010 & +13 & 240389 & 267209 & +11 \\
& 7,624 & 432986 & 494658 & +14 & 1193896 & 1410628 & +18\\
\bottomrule
\end{tabular}
\end{table}


\begin{figure}[h]
    \begin{subfigure}{0.5\textwidth}
    \centering
    \setlength\figurewidth{0.8\linewidth}
    \setlength\figureheight{11em}
    % This file was created by matlab2tikz.
%
\definecolor{mycolor1}{rgb}{0.12527,0.32424,0.83027}%
\definecolor{mycolor2}{rgb}{0.99904,0.76531,0.21641}%
%
\begin{tikzpicture}[%
trim axis left, trim axis right
]

\begin{axis}[%
width=0.96\figurewidth,
height=\figureheight,
at={(0\figurewidth,0\figureheight)},
scale only axis,
bar shift auto,
xmin=0,
xmax=7,
xtick={1,2,3,4,5,6},
xticklabels={{5,15},{5,37},{6,13},{6,44},{7,57},{7,84}},
xticklabel style = {font=\footnotesize},
xlabel style={font=\color{white!15!black}},
xlabel={$f_r$ (\unit{GHz})},
ymin=0,
ymax=3e-07,
ylabel style={font=\color{white!15!black}},
ylabel={$1/Q_o$},
yticklabel style={font=\footnotesize},
axis background/.style={fill=white},
title style={font=\bfseries},
title={},
legend style={legend cell align=left, align=left, draw=white!15!black,legend pos = north west}
]
\addplot[ybar, bar width=0.229, fill=mycolor1, draw=black, area legend] table[row sep=crcr] {%
1	6.30549618735363e-08\\
2	7.01023767596739e-08\\
3	6.34400611286079e-08\\
4	3.77130813762326e-08\\
5	1.15668525313106e-07\\
6	2.52305686010039e-07\\
};
\addplot[forget plot, color=white!15!black] table[row sep=crcr] {%
0	0\\
7	0\\
};
\addlegendentry{utan filter}

\addplot[ybar, bar width=0.229, fill=mycolor2, draw=black, area legend] table[row sep=crcr] {%
1	4.08124644535775e-08\\
2	4.11019336724848e-08\\
3	1.42139113974824e-08\\
4	4.26629500372091e-08\\
5	1.75105525406755e-07\\
6	2.1613786228122e-07\\
};
\addplot[forget plot, color=white!15!black] table[row sep=crcr] {%
0	0\\
7	0\\
};
\addlegendentry{med filter}

\end{axis}
\end{tikzpicture}%
    \caption{Prov 1}
    \label{fig:Q_other_C}
    \end{subfigure}
    \begin{subfigure}{0.5\textwidth}
    \centering
    \setlength\figurewidth{0.8\linewidth}
    \setlength\figureheight{11em}
    % This file was created by matlab2tikz.
%
\definecolor{mycolor1}{rgb}{0.12527,0.32424,0.83027}%
\definecolor{mycolor2}{rgb}{0.99904,0.76531,0.21641}%
%
\begin{tikzpicture}[%
trim axis left, trim axis right
]

\begin{axis}[%
width=0.96\figurewidth,
height=\figureheight,
at={(0\figurewidth,0\figureheight)},
scale only axis,
bar shift auto,
xmin=0,
xmax=7,
xtick={1,2,3,4,5,6},
xticklabels={{4,82},{5,01},{5,73},{6,03},{7,08},{7,32}},
xticklabel style={font=\footnotesize},
xlabel style={font=\color{white!15!black}},
xlabel={$f_r$ (\unit{GHz})},
ymin=0,
ymax=1.5e-06,
ylabel style={font=\color{white!15!black}},
ylabel={$1/Q_o$},
yticklabel style={font=\footnotesize},
axis background/.style={fill=white},
title style={font=\bfseries},
title={},
legend style={at={(0.03,0.97)}, anchor=north west, legend cell align=left, align=left, draw=white!15!black}
]
\addplot[ybar, bar width=0.229, fill=mycolor1, draw=black, area legend] table[row sep=crcr] {%
1	4.58419825830351e-07\\
2	4.91937737970534e-07\\
3	5.54169809853124e-07\\
4	4.62732133455839e-07\\
5	6.74115628511589e-07\\
6	8.77602589651056e-07\\
};
\addplot[forget plot, color=white!15!black] table[row sep=crcr] {%
0	0\\
7	0\\
};
\addlegendentry{utan filter}

\addplot[ybar, bar width=0.229, fill=mycolor2, draw=black, area legend] table[row sep=crcr] {%
1	3.04198848676838e-07\\
2	3.50049432143065e-07\\
3	3.28035188151382e-07\\
4	2.87052605969869e-07\\
5	4.74508917723777e-07\\
6	1.46200417792788e-06\\
};
\addplot[forget plot, color=white!15!black] table[row sep=crcr] {%
0	0\\
7	0\\
};
\addlegendentry{med filter}

\end{axis}
\end{tikzpicture}%
    \caption{Prov 2}
    \label{fig:Q_other_d}
    \end{subfigure}
    \begin{center}
    \begin{subfigure}{0.75\textwidth}
    \centering
    \setlength\figurewidth{0.8\linewidth}
    \setlength\figureheight{11em}
    % This file was created by matlab2tikz.
%
\definecolor{mycolor1}{rgb}{0.12527,0.32424,0.83027}%
\definecolor{mycolor2}{rgb}{0.99904,0.76531,0.21641}%
%
\begin{tikzpicture}[%
trim axis left, trim axis right
]

\begin{axis}[%
width=0.96\figurewidth,
height=\figureheight,
at={(0\figurewidth,0\figureheight)},
scale only axis,
bar shift auto,
xmin=0,
xmax=10,
xtick={1,2,3,4,5,6,7,8,9},
xticklabels={{4.953},{5.006},{5.218},{5.689},{5.959},{6.251},{6.946},{7.351},{7.624}},
xlabel style={font=\color{white!15!black}},
xlabel={Resonansfrekvens (\unit{GHz})},
ymin=0,
ymax=1e-05,
ylabel style={font=\color{white!15!black}},
ylabel={$1/Q_o$},
axis background/.style={fill=white},
title style={font=\bfseries},
title={},
axis x line*=bottom,
axis y line*=left,
legend style={legend cell align=left, align=left, draw=white!15!black}
]
\addplot[ybar, bar width=0.229, fill=mycolor1, draw=black, area legend] table[row sep=crcr] {%
1	4.17784145538606e-07\\
2	5.54828668926099e-06\\
3	6.32830955730862e-07\\
4	2.93833437172275e-06\\
5	1.69718196196557e-06\\
8	2.53328393751367e-06\\
6	8.82972087457609e-06\\
7	4.03277602782519e-06\\
9	5.07823900963456e-07\\
};
\addplot[forget plot, color=white!15!black] table[row sep=crcr] {%
0	0\\
10	0\\
};
\addlegendentry{utan filter}

\addplot[ybar, bar width=0.229, fill=mycolor2, draw=black, area legend] table[row sep=crcr] {%
1	4.26920066540144e-07\\
2	5.33980650200332e-06\\
3	7.03137022021399e-07\\
4	3.91309025572688e-06\\
5	1.51479763666507e-06\\
8	2.78274475976737e-06\\
6	9.50500678222247e-06\\
7	3.51241501320627e-06\\
9	3.46547848189305e-07\\
};
\addplot[forget plot, color=white!15!black] table[row sep=crcr] {%
0	0\\
10	0\\
};
\addlegendentry{med filter}

\end{axis}
\end{tikzpicture}%
    \caption{Prov 3}
    \label{fig:Q_other_A}
    \end{subfigure}
    \end{center}
    \caption{Anpassningsparameter $\frac{1}{Q_o}$ i TLS-modell \eqref{ekv:TLSmodel} anpassad till $Q_i$ mot $\expval{n}$ (exempel se \figref{fig:Qi_n}) för samtliga resonatorer på prov 1 \subref{fig:Q_other_C}, prov 2 \subref{fig:Q_other_d} och prov 3 \subref{fig:Q_other_A}. Till prov 1 användes det långa filtret, prov 2 det medellånga och prov 3 det korta. I prov 1 ser vi en minskning av förlusttermen $\frac{1}{Q_o}$ för fyra av sex resonatorer när filter användes, en resonator (\unit[$f_r=6,436$]{GHz}) var relativt oförändrad medan en resonator(\unit[$f_r=7,569$]{GHz}) uppvisade en ökning i $\frac{1}{Q_o}$ när filter användes. För prov 2 minskade $\frac{1}{Q_o}$ för fem av sex resonatorer med filter, dock uppvisade en resonator en kraftig ökning av $\frac{1}{Q_o}$. För prov 3 var filtrets effekt mer blandad. För samtliga resonatorer var förändringar i $\frac{1}{Q_o}$ relativt små. Bland resonatorerna på prov 3 syns inga större förbättringar samt försämringar när filter användes.}
    \label{fig:my_label}
\end{figure}



\end{document}