% CREATED BY DAVID FRISK, 2016

\documentclass[../../main.tex]{subfiles}
 
\begin{document}

\titel\\
\undertitel\\
PHILIP EDENBORG\\LINA HULTQUIST\\MATTIAS SJÖSTEDT\\JOHAN WINTHER\\
\department\\
\university \setlength{\parskip}{0.5cm}

\thispagestyle{plain}			% Supress header 
\setlength{\parskip}{0pt plus 1.0pt}
\section*{Abstract}



We hypothesize that commercially available bandpass filters allows photons that have a frequency above the bandgap frequency at \unit[88]{GHz} for superconducting aluminum to pass through, thus causing losses for superconductiong microwaveresonators. We manufactured distributed low pass filters to raise the internal quality factor $Q_i$ for the resonators. The exposed conductors for the filters were made with varying length and were surrounded by Stycast. The measurements were carried out with and without filters, the data was then fitted to a model to obtain values for $Q_i$. The measurements showed that our filters in some cases contributed to an increment of $Q_i$, at a maximum of $46 \%$. We can draw the conclusion that our filter, after further development, can become an important component for reducing losses if the resonators of aluminium are restricted by quasiparticles.


 %This report concerns a bachelors project, which purpose was to raise the quality factor for superconducting microwaveresonators. The chosen method was to manufacure distributed low pass filters, the length of the exposed connector varied for the different filters that were made. The connectors were surrounded by a dielectric. The filters purpose was to shield the resonators from photons with a frequency above \unit[88]{GHz}, because these high-frequency photons are believed to break Cooperpairs in the resonators, producing quasiparticles that lowers the quality factor. From our measurements, data could be extracted and using a model fitting, the quality factor could be calculated.

This thesis is written in Swedish.

% KEYWORDS (MAXIMUM 10 WORDS)
\vfill
Keywords: superconducting, Cooper pair, resonator, distributed, filter, cryostat

\section*{Sammandrag}
Vi hypotiserar att kommersiella bandpassfilter släpper igenom fotoner över supraledande aluminiums bandgapsfrekvens \unit[88]{GHz} som orsakar förluster i supraledande mikrovågsresonatorer. Vi tillverkade distribuerade lågpassfilter för att öka interna kvalitetsfaktorn $Q_i$ för resonatorerna. Filtrens frilagda ledare var av varierande längd och omgavs av Stycast. Mätningar utfördes med och utan filter, vilket sedan anpassades till en modell för att erhålla värden på $Q_i$. Mätningarna påvisade att våra filter i vissa fall bidrog till en ökning av $Q_i$, maximalt  $\unit[46]{\%}$. Vi drar slutsatsen att våra filter, efter ytterligare vidareutveckling, kan bli en viktig komponent för att reducera förluster om resonatorerna av aluminium begränsas av kvasipartiklar.

\vfill
Keywords: distribuerade lågpassfilter, supraledande mikrovågsresonatorer, kvasipartiklar, Cooperpar, aluminium

\newpage\null\thispagestyle{empty}\newpage

\end{document}