% CREATED BY DAVID FRISK, 2016

\documentclass[../../main.tex]{subfiles}
 
\begin{document}

\titel\\
\undertitel\\
PHILIP EDENBORG\\LINA HULTQUIST\\MATTIAS SJÖSTEDT\\JOHAN WINTHER\\
\department\\
\university \setlength{\parskip}{0.5cm}

\thispagestyle{plain}			% Supress header 
\setlength{\parskip}{0pt plus 1.0pt}
\section*{Abstract}
Lorem ipsum dolor sit amet, consectetur adipisicing elit, sed do eiusmod tempor incididunt ut labore et dolore magna aliqua. Ut enim ad minim veniam, quis nostrud exercitation ullamco laboris nisi ut aliquip ex ea commodo consequat. Duis aute irure dolor in reprehenderit in voluptate velit esse cillum dolore eu fugiat nulla pariatur. Excepteur sint occaecat cupidatat non proident, sunt in culpa qui officia deserunt mollit anim id est laborum.

This thesis is written in Swedish.

% KEYWORDS (MAXIMUM 10 WORDS)
\vfill
Keywords: superconducting, Cooper pair, resonator, distributed, filter, cryostat

\section*{Sammandrag}
Den här rapporten berör ett kandidatarbete vars syfte var att öka $Q$-värdet för supraledande mikrovågsresonatorer. Vi ville göra detta genom att tillverka distribuerade lågpassfilter med varierande längd och skärma resonatorerna från fotoner med frekvenser på över \unit[88]{GHz}. Filtren tillverkades och karakteriserades och $Q$-värdet för flera resonatorer beräknades med och utan dessa filter. Vi kunde uppvisa förbättringar av $Q_i$ upp till $46 \%$. Vi drar slutsatsen att några av våra filter minskade antalet fotoner bla bla hjälp tack. Filtrena skulle kunna förbättras genom att att avända ett annat dielektrikum.

% KEYWORDS (MAXIMUM 10 WORDS)
\vfill
Keywords: supraledning, Cooper-par, resonator, distribuerade, filter, kryostat

\newpage\null\thispagestyle{empty}\newpage

\end{document}