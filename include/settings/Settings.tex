% CREATED BY DAVID FRISK, 2016

% BASIC SETTINGS
\usepackage{moreverb}								% List settings
\usepackage{textcomp}								% Fonts, symbols etc.
\usepackage{lmodern}								% Latin modern font
\usepackage{helvet}									% Enables font switching
\usepackage[T1]{fontenc}							% Output settings
\usepackage[swedish]{babel}							% Language settings
\usepackage[utf8]{inputenc}							% Input settings
\usepackage{amsmath}								% Mathematical expressions (American mathematical society)
\usepackage{amssymb}								% Mathematical symbols (American mathematical society)
\usepackage{graphicx}								% Figures
%\usepackage{subfig}									% Enables subfigures
\usepackage[subrefformat=parens]{subcaption} 
\numberwithin{equation}{chapter}					% Numbering order for equations
\numberwithin{figure}{chapter}						% Numbering order for figures
\numberwithin{table}{chapter}						% Numbering order for tables
\usepackage{listings}								% Enables source code listings
%\usepackage{chemfig}								% Chemical structures
\usepackage[top=2cm, bottom=2cm,
			inner=2cm, outer=2cm]{geometry}			% Page margin lengths			
\usepackage{eso-pic}								% Create cover page background
\newcommand{\backgroundpic}[3]{
	\put(#1,#2){
	\parbox[b][\paperheight]{\paperwidth}{
	\centering
	\includegraphics[width=\paperwidth,height=\paperheight,keepaspectratio]{#3}}}}
\usepackage{float} 									% Enables object position enforcement using [H]
\usepackage{parskip}								% Enables vertical spaces correctly 
\usepackage{setspace}

\graphicspath{{/}{../}}
\usepackage{subfiles}
\usepackage{units}
\usepackage{lmodern}
%\usepackage{todonotes}
%\usepackage{float}
%\usepackage{units}
\usepackage{icomma}
\usepackage{color}
\usepackage{pgfplots}
\pgfplotsset{compat=newest} 
\pgfplotsset{plot coordinates/math parser=false}
\pgfplotsset{/pgf/number format/use comma}
%\usepackage{braket}
\usepackage{bbm}
%\usepackage{adjustbox}
%\usepackage{placeins}
\usepackage{wrapfig}
\usepackage{marvosym}
\usepackage{dsfont}
%\usepackage[retainorgcmds]{IEEEtrantools}
\usepackage{upgreek}
\usepackage{pdfpages}
\usepackage{booktabs}
\usepackage{url}
\usepackage{physics}
\usepackage{csquotes}
\usepackage[americanresistors,americaninductors]{circuitikz}
\usepackage{xstring}

\def\changemargin#1#2{\list{}{\rightmargin#2\leftmargin#1}\item[]}
\let\endchangemargin=\endlist 


\usepackage[style=ieee,maxcitenames=1,mincitenames=1,maxbibnames=2,minbibnames=2,backend=biber]{biblatex}
\addbibresource{ref.bib}
% \makeatletter

% \newrobustcmd*{\nobibliography}{%
%   \@ifnextchar[%]
%     {\blx@nobibliography}
%     {\blx@nobibliography[]}}

% \def\blx@nobibliography[#1]{}

% \appto{\skip@preamble}{\let\printbibliography\nobibliography}

% \makeatother

\DefineBibliographyStrings{swedish}{%
  andothers = {et\addabbrvspace al\adddot}
}



\usepackage{multirow}

% OPTIONAL SETTINGS (DELETE OR COMMENT TO SUPRESS)

% Disable automatic indentation (equal to using \noindent)
\setlength{\parindent}{0cm}


% Caption settings (aligned left with bold name)
\usepackage[labelfont=bf, textfont=normal,
			justification=justified,
			singlelinecheck=false]{caption} 		

		  	
% Activate clickable links in table of contents  	
\usepackage{hyperref}								
\hypersetup{colorlinks, citecolor=black,
   		 	filecolor=black, linkcolor=black,
    		urlcolor=black}


% Define the number of section levels to be included in the t.o.c. and numbered	(3 is default)	
\setcounter{tocdepth}{5}							
\setcounter{secnumdepth}{5}	


% Chapter title settings
\usepackage{titlesec}		
\titleformat{\chapter}[display]
  {\huge\bfseries\filcenter}
  {{\fontsize{30pt}{1em}\vspace{-7.2ex}\selectfont \textnormal{\thechapter}}}{1ex}{\vspace{-0.5cm}}[\vspace{-1cm}]


% Header and footer settings (Select TWOSIDE or ONESIDE layout below)
\usepackage{fancyhdr}								
\pagestyle{fancy}  
\renewcommand{\chaptermark}[1]{\markboth{\thechapter.\space#1}{}} 


% Select one-sided (1) or two-sided (2) page numbering
\def\layout{1}	% Choose 1 for one-sided or 2 for two-sided layout
% Conditional expression based on the layout choice
\ifnum\layout=2	% Two-sided
    \fancyhf{}			 						
	\fancyhead[LE,RO]{\nouppercase{ \leftmark}}
	\fancyfoot[LE,RO]{\thepage}
	\fancypagestyle{plain}{			% Redefine the plain page style
	\fancyhf{}
	\renewcommand{\headrulewidth}{0pt} 		
	\fancyfoot[LE,RO]{\thepage}}	
\else			% One-sided  	
  	\fancyhf{}					
	\fancyhead[C]{\nouppercase{ \leftmark}}
	\fancyfoot[C]{\thepage}
\fi


% Enable To-do notes
\usepackage[textsize=tiny]{todonotes}   % Include the option "disable" to hide all notes
\setlength{\marginparwidth}{2.5cm} 


% Supress warning from Texmaker about headheight
\setlength{\headheight}{15pt}		

% Våra kommandon som fanns i planeringsrapporten

\newcommand{\email}[1]{\href{mailto:#1}{\texttt{#1}}}
\newcommand{\Figref}[1]{\figurename~\ref{#1}}
\newcommand{\figref}[1]{Figur~\ref{#1}}

\newcommand{\titel}{Lågpassfilter för supraledande mikrovågsresonatorer}
\newcommand{\undertitel}{För minimering av kvasipartikelrelaterade förluster från fotoner över \unit[88]{GHz}, f}
\newcommand{\department}{Institutionen för mikroteknologi och nanovetenskap}
\newcommand{\division}{Avdelningen för kvantkomponentfysik}
\newcommand{\university}{Chalmers tekniska högskola}
\newcommand{\adress}{Göteborg, Sverige \the\year}
\newcommand{\reportnr}{Kandidatarbete MCCX02-17-10}

% Figurinställningar
\newlength\figureheight
\newlength\figurewidth

\newcommand{\tikzfig}[5]{
\begin{figure}[H]
\centering
	\setlength\figureheight{#5}
	\setlength\figurewidth{#4}
	\input{figure/#1.tikz}
    \caption{#2}
    \label{#3}
\end{figure}
}

%Skapar centerfloat som används för att tvinga bilder till att lägga sig i mitten oavsett marginaler. Använd istället för \centering
\makeatletter
\newcommand*{\centerfloat}{%
  \parindent \z@
  \leftskip \z@ \@plus 1fil \@minus \textwidth
  \rightskip\leftskip
  \parfillskip \z@skip}
\makeatother

% Centerar subfig captions
\captionsetup[subfigure]{justification=centering}

\usepackage{calc}
\newsavebox\mybox

% Lokal kompilering
%\usepackage{tikz}
%\usetikzlibrary{external}
%\tikzexternalize % activate!