% CREATED BY DAVID FRISK, 2016
\documentclass[../../main.tex]{subfiles}
 
\begin{document}

\chapter{Allmänna transmissionsledningar}
\label{app:trans}
En generell transmissionsledning kan karakteriseras av de fyra olika parametrarna $R, L, G$ och $C$. Där $R$ är resistans per längdenhet i \unit{$\Omega$/m}, $L$ induktans per längdenhet i \unit{H/m}, $G$ konduktans per längdenhet i \unit{S/m} och $C$ kapacitans per längdenhet i \unit{F/m}. Dessa distribuerade parmetrar kan representeras med den ekvivalent kretsen för en allmän transmissionsledning av längd $\Delta z$ som visas i \figref{fig:transmission_eqv}\autocite{cheng}.


\begin{figure}[H]
\centering
\ctikzset{bipoles/thickness=1}
\begin{circuitikz}
\draw (0,0)
node[anchor=east](in+){+}  
to[short,o-] (1,0)
to[R=$R\Delta z$] (3,0)
to[L=$L\Delta z$,-*] (5,0)
node[anchor=south]{A} 
to[short,-o] (8,0)
node[anchor=west](out+){+} ;

\draw (5,0)
to[short] (5,-1)
to[short] (4,-1)
to[R,l_=$G\Delta z$] (4,-3)
to[short] (5,-3)
to[short] (5,-4);
\draw (5,-1)
to[short] (6,-1)
to[C,l=$C\Delta z$] (6,-3)
to[short] (5,-3);

\draw (0,-4)
node[anchor=east](in-){-} 
to[short,o-o] (8,-4)
node[anchor=west](out-){-} ;

\draw[draw=none](in+)--(in-)node[midway,anchor=east]{$V(z)$};
\draw[draw=none](out+)--(out-)node[midway,anchor=west]{$V(z+\Delta z)$};

[line width=1pt,scale=0.75]
\end{circuitikz}
\caption{Ekvivalent kretsschema för allmän transmissionsledning av längd $\Delta z$.}
\label{fig:transmission_eqv}
\end{figure}



Från den ekvivalenta kretsen i \figref{fig:transmission_eqv} kan vi med hjälp av Kirchoffs strömlag applicerad i nod A samt spänningslag över $R$ och $L$ relatera spänningen till strömmen och vice versa. Om vi sedan låter $\Delta z\rightarrow 0$ får vi de generella transmissionsledningsekvationerna

% Behöver fixa ekvivalenspilen
\begin{equation}
    \centering
    \begin{split}    
        -\dv{V(z)}{z}&=(R+j\omega L)I(z)\\ -\dv{I(z)}{z}&=(G+j\omega C)V(z)
    \end{split}
    \quad\Leftrightarrow\quad
    \begin{split}
        \dv{^2V(z)}{z^2}&=\gamma^2V(z)\\     \dv{^2I(z)}{z^2}&=\gamma^2I(z)
    \end{split}
\end{equation}

% Borde vara en punkt här någonstans?

Där $\gamma$ är propagationskonstanten för transmissionsledningen och ges av
\begin{equation*}
    \gamma=\alpha +j\beta = \sqrt{(R+j\omega L)(G+j\omega C)} \hspace{2em} (\unit{m^{-1}})
\end{equation*}
där $\alpha$ är en dämpningskonstant med enhet \unit{Np/m} och $\beta$ en faskonstant med enhet \unit{rad/m}.

\end{document}