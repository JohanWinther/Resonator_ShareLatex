% CREATED BY DAVID FRISK, 2016
\documentclass[../../main.tex]{subfiles}
 
\begin{document}
\chapter{Numerisk anpassning av en cirkel}
\label{sec:algebraisk}
Följande appendix beskriver en numerisk anpassning till en cirkel enligt \cite[kap. 4.2]{Chernov2005}.

En cirkel kan parametriseras med följande ekvation och tvång:
\begin{align}
    A(x^2+y^2)+Bx+Cy+D&=0\\
    B^2+C^2-4AD&=1
\end{align}

Med definitionen $z=(x^2+y^2)$ vill vi minimera följande funktion:
\begin{equation}
    F(A,B,C,D)=\sum_{i=1}^n \qty(Az_i+Bx_i+Cy_i+D)^2
\end{equation}
där index $i$ står för varje mätpunkt. Denna kan skrivas på matrisform $F=\mathbf{A}^TM\mathbf{A}$ med $\mathbf{A}=(A,B,C,D)^T$.
Även tvånget kan skrivas på matrisform $\mathbf{A}^TP\mathbf{A}=1$ där
\begin{equation*}
    P=\begin{pmatrix}
    0 & 0 & 0 &-2\\
    0 & 1 & 0 & 0 \\
    0 & 0 & 1 & 0\\
   -2 & 0 & 0 & 0
   \end{pmatrix}.
\end{equation*}

$F$ kan nu minimeras med en Lagrange multiplikator för att ta hänsyn till tvånget
\begin{equation}
    \mathcal{L}(\mathbf{A},\lambda)=\mathbf{A}^TM\mathbf{A}-\lambda(\mathbf{A}^TP\mathbf{A}-1)
    \label{eq:lagrangian}
\end{equation}
och egenvektorn $\mathbf{A}^*$ till det $\lambda$ som minimerar \eqref{eq:lagrangian} beräknas med standardiserade numeriska metoder.

Cirkelns position och diameter kan nu utläsas ur $\mathbf{A}^*$ med följande relationer:
\begin{align}
    x_c&=-\frac{B}{2A}\\
    y_c&=-\frac{C}{2A}\\
    r_0&=\frac{\sqrt{B^2+C^2-4AD}}{2|A|}=\frac{1}{2|A|}.
\end{align}
\end{document}