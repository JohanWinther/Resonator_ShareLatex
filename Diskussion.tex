% CREATED BY DAVID FRISK, 2016
\documentclass[main.tex]{subfiles}
 
\begin{document}
\chapter{Diskussion}
\label{ch:discussion} 



\section{Val av dielektrikum}
Den ursprungliga planen var att använda Eccosorb som dielektrikum i filtrena. Det var dock inte möjligt att få tag på i tid för det här projektet. Förutom de rent dissipativa förlusterna som Stycast gav upphov till i våra filter, ger Eccosorb även upphov till magnetiska förluster som ett resultat av att det är laddat med metallpartiklar\autocite{Eccosorb}. Det är rimligt att ,utifrån vad som står i databladet för Eccosorb, anta att dessa frekvensberoende förluster skulle ge oss en skarpare dämpning vid frekvenser på \unit[8-50]{GHz} än vad som syns i figur \ref{fig:batch2_long}. Exempelvis gäller för den vanliga Eccosorb-typen CR-116 att den ger en dämpning på ca \unit[57]{dB/cm} vid \unit[18]{GHz}\autocite{Eccosorb} vilket för våra långa filter blir en dämpning på cirka \unit[150]{dB} jämfört med en dämpning på cirka \unit[10]{dB} vid samma frekvens för Stycast. Detta skulle troligtvis påverkar Q-faktorn positivt för effekter då hypotesen om sönderslagna Cooper-par verkar stämma.



\section{Mätosäkerhet}
För samtliga mätningar har vi tagit hänsyn till en uppskattad dämpning hos våra filter. Det finns dock en viss osäkerhet när det kommer till filterdämpningens påverkan eftersom filtrena är karakteriserade i rumstemperatur och inte vid \unit[10]{mK} som mätningarna på resonatorerna är utförda vid.


\section{Modellanpassning}
När passar TLS-modellen? 

\section{Funktion hos filter}
% hur funkade filter? kan man dra slutsatser
Slusatser om filters funktion...

-När funkar det eller inte + varför


\section{Vidarutveckling}
På grund av tidsbrist hann vi inte mäta med olika filter på samma provlådor. Därför är resultaten svåra att tolka eftersom den stora variationen kan bero på den inbördes skillnaden mellan provlådorna med resonatorerna. Idealt bör minst 10 filterlängder mätas på samma provlåda med förväntan om en genomgående och tydlig trend.

\end{document}