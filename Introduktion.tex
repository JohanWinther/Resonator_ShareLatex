\documentclass[main.tex]{subfiles}

\begin{document}


\chapter{Introduktion}
\section{Bakgrund}
Kvantdatorer är ett aktuellt ämne inom modern forskning, och framstegen har varit markanta de senaste tio åren \autocite{Bylander2017}. Kvantdatorer och kvantsimulatorer i allmänhet har flertalet användningsområden och kan lösa problem som idag inte är möjliga att lösa ens med superdatorer. Tillämpningsområdena finns inom till exempel atomfysik, högenergifysik, kemi, kosmologi. \cite{applications}.


Teknologin bakom kvantsimulering bygger på kvantbitar. Till skillnad från vanliga bitar som enbart kan vara i tillståndet ''0'' eller ''1'' kan en kvantbit befinna sig i en superposition av kvanttillstånden $\ket{0}$ och $\ket{1}$. Denna egenskap för kvantbitarna bidrar till att kvantsimuleringar kan vara exponentiellt mycket kraftfullare än klassiska simuleringar \cite{Eckstein2013}.

Dessa kvantbitar kan realiseras i verkligheten på olika sätt, bland annat med hjälp av supraledande kretsar \cite{Oliver2013}. För att kvantbitarna ska vara användbara är det viktigaste att fokusera på koherenstiden, tiden då en kvantbit kan lagra information. För att kunna kunna utföra beräkningar måste koherenstiden vara tillräckligt hög för att informationen inte ska hinna försvinna innan beräkningarna är utförda\cite{Eckstein2013}. 
 

Eckstein et al. skrev 2013 om materialens inverkan på avfasning och således koherenstid, där det framkom att system som är svåra att koppla internt har mycket längre koherenstid är system som är enkla att koppla \cite{Eckstein2013}. Störta problemen som uppkommer är att informationen ska lagras, sammanflätas under kontroll och avläsas.

2010 publicerade Oliver och Welander en artikel angående materialbekymmer när det kommer till kvantbitar och således även mikrovågsresonatorer \cite{Oliver2013
}. De problem som för tillfället har störst inverkan på koherenstiden kommer till störst del från två-nivåsystem, vilket diskuteras mer djupgående i Sektion \ref{sec:tlf}.  Under de senaste 15 åren har koherenstiden ökat med över en ordning av $10^5$, men möjliga andra bekymmer kan vara kvasipartikel-tunnling och strömfluktuationer som uppkommer i materialen \cite{Oliver2013}.

I en artikel utgiven 2016 skriver Chiaro et al om hur minskade magnetiska förluster till följd av flödesfångande hål i SCPW-resonatorers film (Superconducting Coplanar Waveguide) bidrar till ökade dielektriska ytförluster, här användes en experimentuppställen som var snarlik den som användes i denna rapport. Probem som kan uppkomma är förbättringar på resonatorernas geometriska design som leder till begränsande, dielektriska TLS-förluster. (Fyll på)


För att undersöka hur man kan förbättra koherenstiden för kvantbitar använde vi oss av supraledande resonatorer. Dels för att dessa även är användbara för applikationer tillsammans med kvantbitar, till exempel för avläsning, koppling och filtrering \cite{placeholder} och dels för att de är enklare att arbeta med samtidigt som störningarna som förändrar koherenstiden bör vara de samma\cite{Oliver2013}. Koherenstiden för resonatorer är relaterat till Q-värdet, ett mått på kvalitetsfaktorn för resonatorerna. En högre koherenstid ger ett högre Q-värde.


Målet med detta arbete är att undersöka möjligheter för att öka Q-värdet för supraledande mikrovågsresonatorer. Vår hypotes var att en del av förlusterna uppkommer till följd av att fotoner med frekvenser över \unit[100]{GHz} passerar genom de kommersiella lågpassfiltrena där de sedan slår sönder Cooperpar hos de supraledande resonatorerna. Genom att tillverka distribuerade lågpassfilter som inte släpper igenom dessa högfrekventa fotoner var målet att öka koherenstiden och således öka Q-värdet.




%Våra lågpassfilter skapades genom att placera två cylindriska SMA-kontakter med delvis isolerad centerledare i ett rektangulärt block som fylldes med Stycast. Filtren kopplades sedan till resonatorer i en kryostat med temperatur kring \unit[10]{mK}.
%Genom att bygga filter och sedan mäta på hur väl resonatorerna lagrar energi vid olika frekvenser, när mätningar utfördes med och utan dessa filter, önskade vi höja resonatorernas Q-värde. Samtidigt skulle filtrena konstrueras så att det inte blev för stor dämpning i frekvensbandet \unit[4-8]{GHz} och utan att introducera för stora impedansmissmatchningar i systemet.

Kort introduktion av problemet
fotoner över \unit[88]{GHz} bryter cooperparen i supraledande aluminium

, det är det här vi är intresserade av att ''lösa''

\section{Problembeskrivning}
Komponenter svartkroppstrålning

Hypotes, filter funkar inte, släpper genom strålning bryter par




\section{Syfte}
Syftet med rapporten är att utveckla 

%I nästa kapitel behandlas det som berör resonatorerna, därefter förklaras filterkonstruktion och filterkarakteristik i kapitel \ref{ch:filter}. I kapitel \ref{ch:exp} förklaras de experiment som gav resultaten i \ref{ch:results}.

\end{document}