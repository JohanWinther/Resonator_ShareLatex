\documentclass[main.tex]{subfiles}

\begin{document}


\chapter{Introduktion}
Följande avsnitt kommer att redogöra för bakgrunden för projektet, och problembeskrivning samt projektets syfte kommer att presenteras.
\section{Bakgrund}
Kvantdatorer är ett aktuellt ämne inom modern forskning, och framstegen har varit markanta de senaste tio åren \cite{Bylander2017}. Kvantdatorer och kvantsimulatorer i allmänhet har flertalet användningsområden och kan lösa problem som idag inte är möjliga att lösa ens med superdatorer. Tillämpningsområdena finns inom till exempel atomfysik, högenergifysik, kemi, kosmologi. \cite{applications}.

Teknologin bakom kvantsimulering bygger på kvantbitar (engelska qubit). Till skillnad från vanliga bitar som enbart kan vara i tillståndet ''0'' eller ''1'' kan en kvantbit befinna sig i en superposition av kvanttillstånden $\ket{0}$ och $\ket{1}$. Denna egenskap för kvantbitarna bidrar till att kvantsimuleringar kan vara exponentiellt mycket kraftfullare än klassiska simuleringar \cite{Eckstein2013}.

Dessa kvantbitar kan realiseras i verkligheten på olika sätt \cite{Eckstein2013}, bland annat med supraledande kretsar \cite{Oliver2013}. För att dessa supraledande kvantbitar ska vara användbara behöver de uppfylla en rad kriterier \cite{Eckstein2013}, varav ett krav är att ha en tillräckligt hög koherenstid, tiden då en kvantbit är isolerad från externa störningar \cite{Oliver2013}. Sedan 1999 har koherenstiden ökats från ett fåtal nanosekunder till över 100 mikrosekunder \cite{wendin2016}.

Det finns en konsensus inom forskningsområdet att en ökad förståelse för de faktorer som begränsar koherenstiden är det primära målet för att kunna skapa mer avancerade kvantkretsar och system \cite{Oliver2013}. Där har supraledande mikrovågsresonatorer varit betydande för att identifiera ett flertal källor till störningar och förlustmekanismer för supraledande kretsar. Anledning är för att de är lätta att tillverka och undersöka \cite{Oliver2013}. Dessutom är de användbara för applikationer tillsammans med kvantbitar, till exempel som avläsare av kvantbitarnas tillstånd \cite{wendin2016}.

Den förlustmekanism som dominerar är den som primärt begränsar resonatorerna, vilket förklaras i avsnitt \ref{sec:losses}. Därför är det avgörande att angripa alla orsaker till yttre störningar och förluster i resonatorerna. Förbättringar har gjorts genom bland annat val av design \cite{khalil2011,chiaro2016}, material \cite{Goetz2016} och tillverkningsprocess \cite{Bruno2015,Sanberg2012}.

Resonatorerna kan inte heller prestera bättre än vad mät- och kontrolluppställningen tillåter. \citeauthor{Barends2011} et al påvisar att en kryostat med flerstegsskärmning runt de supraledande resonatorerna helt eliminerar de förluster som uppkommer på grund av termisk strålning i kryostaten \cite{Barends2011}. Även om en resonator skärmas enligt \cite{Barends2011}, täcker inte skärmningen in- eller utgångarna i resonatorkretsen. Därför används bandpassfilter för att dämpa den termiska strålningen i själva ledningen till resonatorerna.

\section{Problembeskrivning}
Vi har anledning att tro att den termiska strålningen kan passera genom dessa bandpassfilter eftersom de alltid har ett begränsat stoppband \cite{santavicca2008}.


i kryostat med flera temperatursteg för att minska mängden svartkroppsstrålning från övriga komponenter i stegen med högre temperatur. I den lägsta temperaturen \unit[10]{mK} innan 
Vår hypotes var att en del av förlusterna uppkommer till följd av att fotoner med frekvenser över \unit[88]{GHz} passerar genom de kommersiella lågpassfiltrena där de sedan slår sönder Cooperpar hos de supraledande resonatorerna. Genom att tillverka distribuerade lågpassfilter som inte släpper igenom dessa högfrekventa fotoner var målet att öka koherenstiden och således öka Q-värdet.


De fotoner som har en frekvens på över \unit[88]{GHz} kan bryta Cooperpar i supraledande aluminium och därmed generera ett överskott av kvasipartiklar. Kvasipartiklar orsakar förluster eftersom de till skillnad från Cooperparen upplever en resistivitet i ledaren. 

De kommersiella filterna som finns idag funkar inte då dämpningen avtar för höga frekvenser, vilket är anledningen till att vi konstruerat egna filter. De högfrekventa fotonerna uppkommer inne i kyostaten, även efter att denna sköldats från omgivningen. Detta beror på att alla komponenter fortfarande uppvisar en liten bakgrundsstrålning, även vid låga temperaturer. Vår hypotes är att genom att tillverka filter som framgångsrikt filtrerar bort fotoner med frekvenser över 88 GHz så kommer dekoherens som uppstår från brutna Cooperpar att minksa och således kommer Q-värdet att öka.

\section{Syfte}
Syftet med rapporten är att redogöra för en möjligt utveckling av filter för mikrovågsresonatorer. Läsaren kommer att få en inblick i tillverkningsprocessen för lågpassfilter lämpade att användas i temperaturer kring 10 mK och som filtrerar bort fotoner med en frekvens över 88 GHz. Att presentera skillnaden i Q-värde när filter placerades före och efter resonatorer i en kryostat med temperatur på 10 mK. I rapporten presenteras data för Q-värden framtagna från resultat

%I nästa kapitel behandlas det som berör resonatorerna, därefter förklaras filterkonstruktion och filterkarakteristik i kapitel \ref{ch:filter}. I kapitel \ref{ch:exp} förklaras de experiment som gav resultaten i \ref{ch:results}.






%Våra lågpassfilter skapades genom att placera två cylindriska SMA-kontakter med delvis isolerad centerledare i ett rektangulärt block som fylldes med Stycast. Filtren kopplades sedan till resonatorer i en kryostat med temperatur kring \unit[10]{mK}.
%Genom att bygga filter och sedan mäta på hur väl resonatorerna lagrar energi vid olika frekvenser, när mätningar utfördes med och utan dessa filter, önskade vi höja resonatorernas Q-värde. Samtidigt skulle filtrena konstrueras så att det inte blev för stor dämpning i frekvensbandet \unit[4-8]{GHz} och utan att introducera för stora impedansmissmatchningar i systemet.


%Eckstein et al. skrev 2013 om materialens inverkan på avfasning och således koherenstid, där det framkom att system som är svåra att koppla internt har mycket längre koherenstid är system som är enkla att koppla . Störta problemen som uppkommer är att informationen ska lagras, sammanflätas under kontroll och avläsas.


%2010 publicerade Oliver och Welander en artikel angående materialbekymmer när det kommer till kvantbitar och således även mikrovågsresonatorer \cite{Oliver2013}. De problem som för tillfället har störst inverkan på koherenstiden kommer till störst del från två-nivåsystem, vilket diskuteras mer djupgående i Sektion \ref{sec:tlf}., men möjliga andra bekymmer kan vara kvasipartikel-tunnling och strömfluktuationer som uppkommer i materialen \cite{Oliver2013}.



%I en artikel utgiven 2016 skriver Chiaro et al om hur minskade magnetiska förluster till följd av flödesfångande hål i SCPW-resonatorers film (Superconducting Coplanar Waveguide) bidrar till ökade dielektriska ytförluster, här användes en experimentuppställen som var snarlik den som användes i denna rapport. Probem som kan uppkomma är förbättringar på resonatorernas geometriska design som leder till begränsande, dielektriska TLS-förluster. (Fyll på)


\end{document}